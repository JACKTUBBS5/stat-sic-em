\documentclass[]{article}
\usepackage{lmodern}
\usepackage{amssymb,amsmath}
\usepackage{ifxetex,ifluatex}
\usepackage{fixltx2e} % provides \textsubscript
\ifnum 0\ifxetex 1\fi\ifluatex 1\fi=0 % if pdftex
  \usepackage[T1]{fontenc}
  \usepackage[utf8]{inputenc}
\else % if luatex or xelatex
  \ifxetex
    \usepackage{mathspec}
  \else
    \usepackage{fontspec}
  \fi
  \defaultfontfeatures{Ligatures=TeX,Scale=MatchLowercase}
\fi
% use upquote if available, for straight quotes in verbatim environments
\IfFileExists{upquote.sty}{\usepackage{upquote}}{}
% use microtype if available
\IfFileExists{microtype.sty}{%
\usepackage[]{microtype}
\UseMicrotypeSet[protrusion]{basicmath} % disable protrusion for tt fonts
}{}
\PassOptionsToPackage{hyphens}{url} % url is loaded by hyperref
\usepackage[unicode=true]{hyperref}
\hypersetup{
            pdftitle={Physical Measures Project},
            pdfauthor={Randy Nguyen, Emmie Jenkins, Sonish Lamsal},
            pdfborder={0 0 0},
            breaklinks=true}
\urlstyle{same}  % don't use monospace font for urls
\usepackage[margin=1in]{geometry}
\usepackage{color}
\usepackage{fancyvrb}
\newcommand{\VerbBar}{|}
\newcommand{\VERB}{\Verb[commandchars=\\\{\}]}
\DefineVerbatimEnvironment{Highlighting}{Verbatim}{commandchars=\\\{\}}
% Add ',fontsize=\small' for more characters per line
\usepackage{framed}
\definecolor{shadecolor}{RGB}{248,248,248}
\newenvironment{Shaded}{\begin{snugshade}}{\end{snugshade}}
\newcommand{\KeywordTok}[1]{\textcolor[rgb]{0.13,0.29,0.53}{\textbf{#1}}}
\newcommand{\DataTypeTok}[1]{\textcolor[rgb]{0.13,0.29,0.53}{#1}}
\newcommand{\DecValTok}[1]{\textcolor[rgb]{0.00,0.00,0.81}{#1}}
\newcommand{\BaseNTok}[1]{\textcolor[rgb]{0.00,0.00,0.81}{#1}}
\newcommand{\FloatTok}[1]{\textcolor[rgb]{0.00,0.00,0.81}{#1}}
\newcommand{\ConstantTok}[1]{\textcolor[rgb]{0.00,0.00,0.00}{#1}}
\newcommand{\CharTok}[1]{\textcolor[rgb]{0.31,0.60,0.02}{#1}}
\newcommand{\SpecialCharTok}[1]{\textcolor[rgb]{0.00,0.00,0.00}{#1}}
\newcommand{\StringTok}[1]{\textcolor[rgb]{0.31,0.60,0.02}{#1}}
\newcommand{\VerbatimStringTok}[1]{\textcolor[rgb]{0.31,0.60,0.02}{#1}}
\newcommand{\SpecialStringTok}[1]{\textcolor[rgb]{0.31,0.60,0.02}{#1}}
\newcommand{\ImportTok}[1]{#1}
\newcommand{\CommentTok}[1]{\textcolor[rgb]{0.56,0.35,0.01}{\textit{#1}}}
\newcommand{\DocumentationTok}[1]{\textcolor[rgb]{0.56,0.35,0.01}{\textbf{\textit{#1}}}}
\newcommand{\AnnotationTok}[1]{\textcolor[rgb]{0.56,0.35,0.01}{\textbf{\textit{#1}}}}
\newcommand{\CommentVarTok}[1]{\textcolor[rgb]{0.56,0.35,0.01}{\textbf{\textit{#1}}}}
\newcommand{\OtherTok}[1]{\textcolor[rgb]{0.56,0.35,0.01}{#1}}
\newcommand{\FunctionTok}[1]{\textcolor[rgb]{0.00,0.00,0.00}{#1}}
\newcommand{\VariableTok}[1]{\textcolor[rgb]{0.00,0.00,0.00}{#1}}
\newcommand{\ControlFlowTok}[1]{\textcolor[rgb]{0.13,0.29,0.53}{\textbf{#1}}}
\newcommand{\OperatorTok}[1]{\textcolor[rgb]{0.81,0.36,0.00}{\textbf{#1}}}
\newcommand{\BuiltInTok}[1]{#1}
\newcommand{\ExtensionTok}[1]{#1}
\newcommand{\PreprocessorTok}[1]{\textcolor[rgb]{0.56,0.35,0.01}{\textit{#1}}}
\newcommand{\AttributeTok}[1]{\textcolor[rgb]{0.77,0.63,0.00}{#1}}
\newcommand{\RegionMarkerTok}[1]{#1}
\newcommand{\InformationTok}[1]{\textcolor[rgb]{0.56,0.35,0.01}{\textbf{\textit{#1}}}}
\newcommand{\WarningTok}[1]{\textcolor[rgb]{0.56,0.35,0.01}{\textbf{\textit{#1}}}}
\newcommand{\AlertTok}[1]{\textcolor[rgb]{0.94,0.16,0.16}{#1}}
\newcommand{\ErrorTok}[1]{\textcolor[rgb]{0.64,0.00,0.00}{\textbf{#1}}}
\newcommand{\NormalTok}[1]{#1}
\usepackage{graphicx,grffile}
\makeatletter
\def\maxwidth{\ifdim\Gin@nat@width>\linewidth\linewidth\else\Gin@nat@width\fi}
\def\maxheight{\ifdim\Gin@nat@height>\textheight\textheight\else\Gin@nat@height\fi}
\makeatother
% Scale images if necessary, so that they will not overflow the page
% margins by default, and it is still possible to overwrite the defaults
% using explicit options in \includegraphics[width, height, ...]{}
\setkeys{Gin}{width=\maxwidth,height=\maxheight,keepaspectratio}
\IfFileExists{parskip.sty}{%
\usepackage{parskip}
}{% else
\setlength{\parindent}{0pt}
\setlength{\parskip}{6pt plus 2pt minus 1pt}
}
\setlength{\emergencystretch}{3em}  % prevent overfull lines
\providecommand{\tightlist}{%
  \setlength{\itemsep}{0pt}\setlength{\parskip}{0pt}}
\setcounter{secnumdepth}{0}
% Redefines (sub)paragraphs to behave more like sections
\ifx\paragraph\undefined\else
\let\oldparagraph\paragraph
\renewcommand{\paragraph}[1]{\oldparagraph{#1}\mbox{}}
\fi
\ifx\subparagraph\undefined\else
\let\oldsubparagraph\subparagraph
\renewcommand{\subparagraph}[1]{\oldsubparagraph{#1}\mbox{}}
\fi

% set default figure placement to htbp
\makeatletter
\def\fps@figure{htbp}
\makeatother


\title{Physical Measures Project}
\author{Randy Nguyen, Emmie Jenkins, Sonish Lamsal}
\date{11/19/2020}

\begin{document}
\maketitle

\subsection{Background}\label{background}

Larner (1996) measured the weight and various physical measurements for
22 males aged 16--30. \pause \\
Subjects were randomly chosen volunteers and all were in reasonably good
health.\\
\pause
Subjects were requested to tense each muscle being measured slightly to
ensure measurement consistency.\\
\pause
The question of interest for these data is how weight can best be
predicted from the other measurements.

\subsection{Dataset}\label{dataset}

\begin{Shaded}
\begin{Highlighting}[]
\KeywordTok{library}\NormalTok{(readxl)}
\NormalTok{PhysicalMeasures <-}\StringTok{ }\KeywordTok{read_xlsx}\NormalTok{(}\KeywordTok{file.path}\NormalTok{(}\StringTok{"physicalmeasures.xlsx"}\NormalTok{))}
\end{Highlighting}
\end{Shaded}

\subsection{Scatterplot Matrix}\label{scatterplot-matrix}

\begin{Shaded}
\begin{Highlighting}[]
\KeywordTok{library}\NormalTok{(GGally)}
\KeywordTok{ggpairs}\NormalTok{(PhysicalMeasures, }\DataTypeTok{axisLabels =} \StringTok{"none"}\NormalTok{, }
        \DataTypeTok{title =} \StringTok{"Scatterplot Matrix of Physical Measures"}\NormalTok{)}
\end{Highlighting}
\end{Shaded}

\begin{Shaded}
\begin{Highlighting}[]
\CommentTok{# corr codes}
\end{Highlighting}
\end{Shaded}

\subsection{Scatterplot Matrix}\label{scatterplot-matrix-1}

\includegraphics{physicalmeasures_files/figure-latex/scatterplot matrix-1.pdf}

\subsection{Linear Model of Mass with All
Predictors}\label{linear-model-of-mass-with-all-predictors}

\begin{Shaded}
\begin{Highlighting}[]
\NormalTok{fit <-}\StringTok{ }\KeywordTok{lm}\NormalTok{(Mass }\OperatorTok{~}\StringTok{ }\NormalTok{., }\DataTypeTok{data =}\NormalTok{ PhysicalMeasures)}
\KeywordTok{summary}\NormalTok{(fit) }\CommentTok{# does not have VIFs included like SAS}
\end{Highlighting}
\end{Shaded}

\subsection{Linear Model of Mass with All
Predictors}\label{linear-model-of-mass-with-all-predictors-1}

\begin{verbatim}
## 
## Call:
## lm(formula = Mass ~ ., data = PhysicalMeasures)
## 
## Residuals:
##     Min      1Q  Median      3Q     Max 
## -2.5523 -0.9965  0.0461  1.0499  4.1719 
## 
## Coefficients:
##              Estimate Std. Error t value Pr(>|t|)    
## (Intercept) -69.51714   29.03739  -2.394 0.035605 *  
## Fore          1.78182    0.85473   2.085 0.061204 .  
## Bicep         0.15509    0.48530   0.320 0.755275    
## Chest         0.18914    0.22583   0.838 0.420132    
## Neck         -0.48184    0.72067  -0.669 0.517537    
## Shoulder     -0.02931    0.23943  -0.122 0.904769    
## Waist         0.66144    0.11648   5.679 0.000143 ***
## Height        0.31785    0.13037   2.438 0.032935 *  
## Calf          0.44589    0.41251   1.081 0.302865    
## Thigh         0.29721    0.30510   0.974 0.350917    
## Head         -0.91956    0.52009  -1.768 0.104735    
## ---
## Signif. codes:  0 '***' 0.001 '**' 0.01 '*' 0.05 '.' 0.1 ' ' 1
## 
## Residual standard error: 2.287 on 11 degrees of freedom
## Multiple R-squared:  0.9772, Adjusted R-squared:  0.9565 
## F-statistic: 47.17 on 10 and 11 DF,  p-value: 1.408e-07
\end{verbatim}

\subsection{Checking Regression
Assumptions}\label{checking-regression-assumptions}

Linear regression makes several assumptions about the data, such as :
\pause \\
1) Linearity of the data: The relationship between the predictor (x) and
the outcome (y) is assumed to be linear.\\
\pause
2) Normality of residuals: The residual errors are assumed to be
normally distributed.\\
\pause
3) Homogeneity of residuals variance: The residuals are assumed to have
a constant variance (homoscedasticity).\\
\pause
4) Independence of residuals error terms.\\
\pause
You should check whether or not these assumptions hold true.

\subsection{Potential Problems}\label{potential-problems}

\begin{enumerate}
\def\labelenumi{\arabic{enumi})}
\tightlist
\item
  Non-linearity of the outcome - predictor relationships \pause
\item
  Heteroscedasticity: Non-constant variance of error terms. \pause
\item
  Presence of influential values in the data that can be: \pause \\
  -Outliers: extreme values in the outcome (y) variable\\
  -High-leverage points: extreme values in the predictors (x) variable
  \pause \\
  All these assumptions and potential problems can be checked by
  producing some diagnostic plots visualizing the residual errors.
\end{enumerate}

\subsection{Fit Diagnostics}\label{fit-diagnostics}

\begin{Shaded}
\begin{Highlighting}[]
\KeywordTok{fitted}\NormalTok{(fit) }\CommentTok{# predicted values}
\end{Highlighting}
\end{Shaded}

\begin{verbatim}
##        1        2        3        4        5        6        7        8 
## 75.41911 84.99916 62.76981 79.39306 80.52127 94.03260 64.76643 69.38366 
##        9       10       11       12       13       14       15       16 
## 63.41676 60.55234 70.42209 72.48721 75.64616 70.10537 82.32972 66.83032 
##       17       18       19       20       21       22 
## 53.62109 64.89798 83.87527 70.59230 84.82812 95.61014
\end{verbatim}

\subsection{Fit Diagnostics}\label{fit-diagnostics-1}

\begin{Shaded}
\begin{Highlighting}[]
\KeywordTok{residuals}\NormalTok{(fit) }\CommentTok{# residuals}
\end{Highlighting}
\end{Shaded}

\begin{verbatim}
##           1           2           3           4           5           6 
##  1.58089186  0.50083884  0.23019091  1.10694030 -1.02127459 -0.03259848 
##           7           8           9          10          11          12 
##  1.23356577 -0.38366451  1.58324300 -2.55234445 -0.92209230  0.51278557 
##          13          14          15          16          17          18 
## -1.64616130 -2.10536813 -2.32971607 -0.83032498  0.87890675 -0.89798085 
##          19          20          21          22 
##  0.12472670  2.40769818  4.17188248 -1.61014467
\end{verbatim}

\subsection{Fit Diagnostics}\label{fit-diagnostics-2}

\begin{Shaded}
\begin{Highlighting}[]
\KeywordTok{library}\NormalTok{(broom)}
\KeywordTok{options}\NormalTok{(}\DataTypeTok{width =} \DecValTok{80}\NormalTok{)}
\NormalTok{diagnostics <-}\StringTok{ }\KeywordTok{augment}\NormalTok{(fit)}
\NormalTok{diagnostics[, }\DecValTok{12}\OperatorTok{:}\DecValTok{16}\NormalTok{]}
\end{Highlighting}
\end{Shaded}

\begin{verbatim}
## # A tibble: 22 x 5
##    .fitted  .resid .std.resid  .hat .sigma
##      <dbl>   <dbl>      <dbl> <dbl>  <dbl>
##  1    75.4  1.58       0.837  0.318   2.32
##  2    85.0  0.501      0.264  0.314   2.39
##  3    62.8  0.230      0.141  0.489   2.40
##  4    79.4  1.11       0.563  0.262   2.36
##  5    80.5 -1.02      -0.539  0.314   2.37
##  6    94.0 -0.0326    -0.0251 0.678   2.40
##  7    64.8  1.23       0.845  0.593   2.32
##  8    69.4 -0.384     -0.213  0.382   2.39
##  9    63.4  1.58       1.21   0.675   2.23
## 10    60.6 -2.55      -1.56   0.490   2.12
## # ... with 12 more rows
\end{verbatim}

\subsection{Fit Diagnostics}\label{fit-diagnostics-3}

\begin{Shaded}
\begin{Highlighting}[]
\KeywordTok{anova}\NormalTok{(fit) }
\end{Highlighting}
\end{Shaded}

\begin{verbatim}
## Analysis of Variance Table
## 
## Response: Mass
##           Df  Sum Sq Mean Sq  F value    Pr(>F)    
## Fore       1 2038.88 2038.88 389.8866 6.131e-10 ***
## Bicep      1    0.13    0.13   0.0248  0.877633    
## Chest      1   79.61   79.61  15.2237  0.002469 ** 
## Neck       1   23.83   23.83   4.5567  0.056126 .  
## Shoulder   1    5.51    5.51   1.0540  0.326643    
## Waist      1  237.66  237.66  45.4464 3.194e-05 ***
## Height     1   37.91   37.91   7.2489  0.020940 *  
## Calf       1   12.80   12.80   2.4478  0.145987    
## Thigh      1   13.95   13.95   2.6669  0.130723    
## Head       1   16.35   16.35   3.1261  0.104735    
## Residuals 11   57.52    5.23                       
## ---
## Signif. codes:  0 '***' 0.001 '**' 0.01 '*' 0.05 '.' 0.1 ' ' 1
\end{verbatim}

\subsection{Fit Diagnostics}\label{fit-diagnostics-4}

\begin{Shaded}
\begin{Highlighting}[]
\KeywordTok{plot}\NormalTok{(fit, }\DataTypeTok{which =} \DecValTok{1}\NormalTok{)}
\end{Highlighting}
\end{Shaded}

\includegraphics{physicalmeasures_files/figure-latex/diagnostic plot 1-1.pdf}

\subsection{Fit Diagnostics}\label{fit-diagnostics-5}

\begin{Shaded}
\begin{Highlighting}[]
\KeywordTok{plot}\NormalTok{(fit, }\DataTypeTok{which =} \DecValTok{2}\NormalTok{)}
\end{Highlighting}
\end{Shaded}

\includegraphics{physicalmeasures_files/figure-latex/diagnostic plot 2-1.pdf}

\subsection{Fit Diagnostics}\label{fit-diagnostics-6}

\begin{Shaded}
\begin{Highlighting}[]
\KeywordTok{plot}\NormalTok{(fit, }\DataTypeTok{which =} \DecValTok{3}\NormalTok{)}
\end{Highlighting}
\end{Shaded}

\includegraphics{physicalmeasures_files/figure-latex/diagnostic plot 3-1.pdf}

\subsection{Fit Diagnostics}\label{fit-diagnostics-7}

\begin{Shaded}
\begin{Highlighting}[]
\KeywordTok{plot}\NormalTok{(fit, }\DataTypeTok{which =} \DecValTok{4}\NormalTok{)}
\end{Highlighting}
\end{Shaded}

\includegraphics{physicalmeasures_files/figure-latex/diagnostic plot 4-1.pdf}

\subsection{Fit Diagnostics}\label{fit-diagnostics-8}

\begin{Shaded}
\begin{Highlighting}[]
\KeywordTok{plot}\NormalTok{(fit, }\DataTypeTok{which =} \DecValTok{5}\NormalTok{)}
\end{Highlighting}
\end{Shaded}

\includegraphics{physicalmeasures_files/figure-latex/diagnostic plot 5-1.pdf}

\subsection{Fit Diagnostics}\label{fit-diagnostics-9}

\begin{Shaded}
\begin{Highlighting}[]
\KeywordTok{plot}\NormalTok{(fit, }\DataTypeTok{which =} \DecValTok{6}\NormalTok{)}
\end{Highlighting}
\end{Shaded}

\includegraphics{physicalmeasures_files/figure-latex/diagnostic plot 6-1.pdf}

\subsection{Residual by Regressors for
Mass}\label{residual-by-regressors-for-mass}

\begin{Shaded}
\begin{Highlighting}[]
\KeywordTok{ggplot}\NormalTok{(diagnostics) }\OperatorTok{+}
\StringTok{  }\KeywordTok{geom_point}\NormalTok{(}\KeywordTok{aes}\NormalTok{(}\DataTypeTok{x =}\NormalTok{ Fore, }\DataTypeTok{y =}\NormalTok{ .resid)) }\OperatorTok{+}
\StringTok{  }\KeywordTok{geom_hline}\NormalTok{(}\DataTypeTok{yintercept =} \DecValTok{0}\NormalTok{) }\OperatorTok{+}
\StringTok{  }\KeywordTok{ylab}\NormalTok{(}\StringTok{"Residual"}\NormalTok{) }\OperatorTok{+}
\StringTok{  }\KeywordTok{theme_minimal}\NormalTok{()}
\end{Highlighting}
\end{Shaded}

\subsection{Residual by Regressors for
Mass}\label{residual-by-regressors-for-mass-1}

\includegraphics{physicalmeasures_files/figure-latex/fit diagnostics 1-1.pdf}

\subsection{Residual by Regressors for
Mass}\label{residual-by-regressors-for-mass-2}

\begin{Shaded}
\begin{Highlighting}[]
\KeywordTok{ggplot}\NormalTok{(diagnostics) }\OperatorTok{+}
\StringTok{  }\KeywordTok{geom_point}\NormalTok{(}\KeywordTok{aes}\NormalTok{(}\DataTypeTok{x =}\NormalTok{ Bicep, }\DataTypeTok{y =}\NormalTok{ .resid)) }\OperatorTok{+}
\StringTok{  }\KeywordTok{geom_hline}\NormalTok{(}\DataTypeTok{yintercept =} \DecValTok{0}\NormalTok{) }\OperatorTok{+}
\StringTok{  }\KeywordTok{ylab}\NormalTok{(}\StringTok{"Residual"}\NormalTok{) }\OperatorTok{+}
\StringTok{  }\KeywordTok{theme_minimal}\NormalTok{()}
\end{Highlighting}
\end{Shaded}

\subsection{Residual by Regressors for
Mass}\label{residual-by-regressors-for-mass-3}

\includegraphics{physicalmeasures_files/figure-latex/fit diagnostics 2-1.pdf}

\subsection{Residual by Regressors for
Mass}\label{residual-by-regressors-for-mass-4}

\begin{Shaded}
\begin{Highlighting}[]
\KeywordTok{ggplot}\NormalTok{(diagnostics) }\OperatorTok{+}
\StringTok{  }\KeywordTok{geom_point}\NormalTok{(}\KeywordTok{aes}\NormalTok{(}\DataTypeTok{x =}\NormalTok{ Chest, }\DataTypeTok{y =}\NormalTok{ .resid)) }\OperatorTok{+}
\StringTok{  }\KeywordTok{geom_hline}\NormalTok{(}\DataTypeTok{yintercept =} \DecValTok{0}\NormalTok{) }\OperatorTok{+}
\StringTok{  }\KeywordTok{ylab}\NormalTok{(}\StringTok{"Residual"}\NormalTok{) }\OperatorTok{+}
\StringTok{  }\KeywordTok{theme_minimal}\NormalTok{()}
\end{Highlighting}
\end{Shaded}

\subsection{Residual by Regressors for
Mass}\label{residual-by-regressors-for-mass-5}

\includegraphics{physicalmeasures_files/figure-latex/fit diagnostics 3-1.pdf}

\subsection{Residual by Regressors for
Mass}\label{residual-by-regressors-for-mass-6}

\begin{Shaded}
\begin{Highlighting}[]
\KeywordTok{ggplot}\NormalTok{(diagnostics) }\OperatorTok{+}
\StringTok{  }\KeywordTok{geom_point}\NormalTok{(}\KeywordTok{aes}\NormalTok{(}\DataTypeTok{x =}\NormalTok{ Neck, }\DataTypeTok{y =}\NormalTok{ .resid)) }\OperatorTok{+}
\StringTok{  }\KeywordTok{geom_hline}\NormalTok{(}\DataTypeTok{yintercept =} \DecValTok{0}\NormalTok{) }\OperatorTok{+}
\StringTok{  }\KeywordTok{ylab}\NormalTok{(}\StringTok{"Residual"}\NormalTok{) }\OperatorTok{+}
\StringTok{  }\KeywordTok{theme_minimal}\NormalTok{()}
\end{Highlighting}
\end{Shaded}

\subsection{Residual by Regressors for
Mass}\label{residual-by-regressors-for-mass-7}

\includegraphics{physicalmeasures_files/figure-latex/fit diagnostics 4-1.pdf}

\subsection{Residual by Regressors for
Mass}\label{residual-by-regressors-for-mass-8}

\begin{Shaded}
\begin{Highlighting}[]
\KeywordTok{ggplot}\NormalTok{(diagnostics) }\OperatorTok{+}
\StringTok{  }\KeywordTok{geom_point}\NormalTok{(}\KeywordTok{aes}\NormalTok{(}\DataTypeTok{x =}\NormalTok{ Shoulder, }\DataTypeTok{y =}\NormalTok{ .resid)) }\OperatorTok{+}
\StringTok{  }\KeywordTok{geom_hline}\NormalTok{(}\DataTypeTok{yintercept =} \DecValTok{0}\NormalTok{) }\OperatorTok{+}
\StringTok{  }\KeywordTok{ylab}\NormalTok{(}\StringTok{"Residual"}\NormalTok{) }\OperatorTok{+}
\StringTok{  }\KeywordTok{theme_minimal}\NormalTok{()}
\end{Highlighting}
\end{Shaded}

\subsection{Residual by Regressors for
Mass}\label{residual-by-regressors-for-mass-9}

\includegraphics{physicalmeasures_files/figure-latex/fit diagnostics 5-1.pdf}

\subsection{Residual by Regressors for
Mass}\label{residual-by-regressors-for-mass-10}

\begin{Shaded}
\begin{Highlighting}[]
\KeywordTok{ggplot}\NormalTok{(diagnostics) }\OperatorTok{+}
\StringTok{  }\KeywordTok{geom_point}\NormalTok{(}\KeywordTok{aes}\NormalTok{(}\DataTypeTok{x =}\NormalTok{ Waist, }\DataTypeTok{y =}\NormalTok{ .resid)) }\OperatorTok{+}
\StringTok{  }\KeywordTok{geom_hline}\NormalTok{(}\DataTypeTok{yintercept =} \DecValTok{0}\NormalTok{) }\OperatorTok{+}
\StringTok{  }\KeywordTok{ylab}\NormalTok{(}\StringTok{"Residual"}\NormalTok{) }\OperatorTok{+}
\StringTok{  }\KeywordTok{theme_minimal}\NormalTok{()}
\end{Highlighting}
\end{Shaded}

\subsection{Residual by Regressors for
Mass}\label{residual-by-regressors-for-mass-11}

\includegraphics{physicalmeasures_files/figure-latex/fit diagnostics 6-1.pdf}

\subsection{Residual by Regressors for
Mass}\label{residual-by-regressors-for-mass-12}

\begin{Shaded}
\begin{Highlighting}[]
\KeywordTok{ggplot}\NormalTok{(diagnostics) }\OperatorTok{+}
\StringTok{  }\KeywordTok{geom_point}\NormalTok{(}\KeywordTok{aes}\NormalTok{(}\DataTypeTok{x =}\NormalTok{ Height, }\DataTypeTok{y =}\NormalTok{ .resid)) }\OperatorTok{+}
\StringTok{  }\KeywordTok{geom_hline}\NormalTok{(}\DataTypeTok{yintercept =} \DecValTok{0}\NormalTok{) }\OperatorTok{+}
\StringTok{  }\KeywordTok{ylab}\NormalTok{(}\StringTok{"Residual"}\NormalTok{) }\OperatorTok{+}
\StringTok{  }\KeywordTok{theme_minimal}\NormalTok{()}
\end{Highlighting}
\end{Shaded}

\subsection{Residual by Regressors for
Mass}\label{residual-by-regressors-for-mass-13}

\includegraphics{physicalmeasures_files/figure-latex/fit diagnostics 7-1.pdf}

\subsection{Residual by Regressors for
Mass}\label{residual-by-regressors-for-mass-14}

\begin{Shaded}
\begin{Highlighting}[]
\KeywordTok{ggplot}\NormalTok{(diagnostics) }\OperatorTok{+}
\StringTok{  }\KeywordTok{geom_point}\NormalTok{(}\KeywordTok{aes}\NormalTok{(}\DataTypeTok{x =}\NormalTok{ Calf, }\DataTypeTok{y =}\NormalTok{ .resid)) }\OperatorTok{+}
\StringTok{  }\KeywordTok{geom_hline}\NormalTok{(}\DataTypeTok{yintercept =} \DecValTok{0}\NormalTok{) }\OperatorTok{+}
\StringTok{  }\KeywordTok{ylab}\NormalTok{(}\StringTok{"Residual"}\NormalTok{) }\OperatorTok{+}
\StringTok{  }\KeywordTok{theme_minimal}\NormalTok{()}
\end{Highlighting}
\end{Shaded}

\subsection{Residual by Regressors for
Mass}\label{residual-by-regressors-for-mass-15}

\includegraphics{physicalmeasures_files/figure-latex/fit diagnostics 8-1.pdf}

\subsection{Residual by Regressors for
Mass}\label{residual-by-regressors-for-mass-16}

\begin{Shaded}
\begin{Highlighting}[]
\KeywordTok{ggplot}\NormalTok{(diagnostics) }\OperatorTok{+}
\StringTok{  }\KeywordTok{geom_point}\NormalTok{(}\KeywordTok{aes}\NormalTok{(}\DataTypeTok{x =}\NormalTok{ Thigh, }\DataTypeTok{y =}\NormalTok{ .resid)) }\OperatorTok{+}
\StringTok{  }\KeywordTok{geom_hline}\NormalTok{(}\DataTypeTok{yintercept=}\DecValTok{0}\NormalTok{) }\OperatorTok{+}
\StringTok{  }\KeywordTok{ylab}\NormalTok{(}\StringTok{"Residual"}\NormalTok{) }\OperatorTok{+}
\StringTok{  }\KeywordTok{theme_minimal}\NormalTok{()}
\end{Highlighting}
\end{Shaded}

\subsection{Residual by Regressors for
Mass}\label{residual-by-regressors-for-mass-17}

\includegraphics{physicalmeasures_files/figure-latex/fit diagnostics 9-1.pdf}

\subsection{Residual by Regressors for
Mass}\label{residual-by-regressors-for-mass-18}

\begin{Shaded}
\begin{Highlighting}[]
\KeywordTok{ggplot}\NormalTok{(diagnostics) }\OperatorTok{+}
\StringTok{  }\KeywordTok{geom_point}\NormalTok{(}\KeywordTok{aes}\NormalTok{(}\DataTypeTok{x =}\NormalTok{ Head, }\DataTypeTok{y =}\NormalTok{ .resid)) }\OperatorTok{+}
\StringTok{  }\KeywordTok{geom_hline}\NormalTok{(}\DataTypeTok{yintercept =} \DecValTok{0}\NormalTok{) }\OperatorTok{+}
\StringTok{  }\KeywordTok{ylab}\NormalTok{(}\StringTok{"Residual"}\NormalTok{) }\OperatorTok{+}
\StringTok{  }\KeywordTok{theme_minimal}\NormalTok{()}
\end{Highlighting}
\end{Shaded}

\subsection{Residual by Regressors for
Mass}\label{residual-by-regressors-for-mass-19}

\includegraphics{physicalmeasures_files/figure-latex/fit diagnostics 10-1.pdf}

\subsection{Evaluate Collinearity}\label{evaluate-collinearity}

\begin{Shaded}
\begin{Highlighting}[]
\KeywordTok{library}\NormalTok{(car)}
\end{Highlighting}
\end{Shaded}

\begin{verbatim}
## Loading required package: carData
\end{verbatim}

\begin{Shaded}
\begin{Highlighting}[]
\KeywordTok{options}\NormalTok{(}\DataTypeTok{width =} \DecValTok{60}\NormalTok{)}
\KeywordTok{vif}\NormalTok{(fit) }\CommentTok{# variance inflation factors}
\KeywordTok{sqrt}\NormalTok{(}\KeywordTok{vif}\NormalTok{(fit)) }\OperatorTok{>}\StringTok{ }\DecValTok{2} \CommentTok{# Ice Cream Flavor}
\end{Highlighting}
\end{Shaded}

\subsection{Evaluate Collinearity}\label{evaluate-collinearity-1}

\begin{verbatim}
##      Fore     Bicep     Chest      Neck  Shoulder     Waist 
## 10.807904  7.986195  9.285241  7.033350  9.381060  3.310982 
##    Height      Calf     Thigh      Head 
##  2.623565  3.992476  4.830343  1.714583
\end{verbatim}

\begin{verbatim}
##     Fore    Bicep    Chest     Neck Shoulder    Waist 
##     TRUE     TRUE     TRUE     TRUE     TRUE    FALSE 
##   Height     Calf    Thigh     Head 
##    FALSE    FALSE     TRUE    FALSE
\end{verbatim}

\subsection{Test for Autocorrelated
Errors}\label{test-for-autocorrelated-errors}

\begin{Shaded}
\begin{Highlighting}[]
\KeywordTok{durbinWatsonTest}\NormalTok{(fit)}
\end{Highlighting}
\end{Shaded}

\begin{verbatim}
##  lag Autocorrelation D-W Statistic p-value
##    1       0.1432626      1.624958   0.376
##  Alternative hypothesis: rho != 0
\end{verbatim}

\subsection{Global Test of Model
Assumptions}\label{global-test-of-model-assumptions}

\begin{Shaded}
\begin{Highlighting}[]
\CommentTok{#library(gvlma)}
\CommentTok{#gvmodel <- gvlma(fit)}
\CommentTok{#display.gvlmatests(gvmodel)}
\end{Highlighting}
\end{Shaded}

\subsection{Global Test of Model
Assumptions}\label{global-test-of-model-assumptions-1}

\subsection{Backward Stepwise Model Selection Based on
AIC}\label{backward-stepwise-model-selection-based-on-aic}

\begin{Shaded}
\begin{Highlighting}[]
\NormalTok{selectedModel <-}\StringTok{ }\KeywordTok{step}\NormalTok{(}\KeywordTok{lm}\NormalTok{(Mass }\OperatorTok{~}\StringTok{ }\NormalTok{., }\DataTypeTok{data =}\NormalTok{ PhysicalMeasures))}
\KeywordTok{summary}\NormalTok{(selectedModel)}
\end{Highlighting}
\end{Shaded}

\subsection{Backward Stepwise Model Selection Based on
AIC}\label{backward-stepwise-model-selection-based-on-aic-1}

\begin{verbatim}
## 
## Call:
## lm(formula = Mass ~ Fore + Waist + Height + Calf + Thigh + Head, 
##     data = PhysicalMeasures)
## 
## Residuals:
##     Min      1Q  Median      3Q     Max 
## -3.2362 -1.3426 -0.0132  0.9784  4.5197 
## 
## Coefficients:
##              Estimate Std. Error t value Pr(>|t|)    
## (Intercept) -79.72624   23.88925  -3.337  0.00450 ** 
## Fore          1.79485    0.48536   3.698  0.00215 ** 
## Waist         0.65671    0.09719   6.757 6.45e-06 ***
## Height        0.25388    0.08059   3.150  0.00661 ** 
## Calf          0.50718    0.34671   1.463  0.16415    
## Thigh         0.43298    0.22801   1.899  0.07698 .  
## Head         -0.65722    0.38200  -1.720  0.10590    
## ---
## Signif. codes:  
## 0 '***' 0.001 '**' 0.01 '*' 0.05 '.' 0.1 ' ' 1
## 
## Residual standard error: 2.077 on 15 degrees of freedom
## Multiple R-squared:  0.9744, Adjusted R-squared:  0.9641 
## F-statistic:    95 on 6 and 15 DF,  p-value: 4.501e-11
\end{verbatim}

\subsection{Check VIFs for
Multicollinearity}\label{check-vifs-for-multicollinearity}

\begin{Shaded}
\begin{Highlighting}[]
\KeywordTok{print}\NormalTok{(all_vifs <-}\StringTok{ }\NormalTok{car}\OperatorTok{::}\KeywordTok{vif}\NormalTok{(selectedModel))}
\end{Highlighting}
\end{Shaded}

\begin{verbatim}
##     Fore    Waist   Height     Calf    Thigh     Head 
## 4.223848 2.793854 1.215188 3.418191 3.269648 1.121044
\end{verbatim}

\subsection{Recursively select models with VIF \textless{}
4}\label{recursively-select-models-with-vif-4}

\begin{Shaded}
\begin{Highlighting}[]
\NormalTok{signif_all <-}\StringTok{ }\KeywordTok{names}\NormalTok{(all_vifs)}
\ControlFlowTok{while}\NormalTok{(}\KeywordTok{any}\NormalTok{(all_vifs }\OperatorTok{>}\StringTok{ }\DecValTok{4}\NormalTok{))\{}
  \CommentTok{# get the var with max vif}
\NormalTok{  var_with_max_vif <-}\StringTok{ }\KeywordTok{names}\NormalTok{(}\KeywordTok{which}\NormalTok{(all_vifs }\OperatorTok{==}\StringTok{ }\KeywordTok{max}\NormalTok{(all_vifs)))  }
  \CommentTok{# remove}
\NormalTok{  signif_all <-}\StringTok{ }\NormalTok{signif_all[}\OperatorTok{!}\NormalTok{(signif_all) }\OperatorTok\StringTok{ }\NormalTok{var_with_max_vif]  }
  \CommentTok{# new formula}
\NormalTok{  myFormula <-}\StringTok{ }\KeywordTok{as.formula}\NormalTok{(}\KeywordTok{paste}\NormalTok{(}\StringTok{"Mass ~ "}\NormalTok{, }
                             \KeywordTok{paste}\NormalTok{ (signif_all, }\DataTypeTok{collapse=}\StringTok{" + "}\NormalTok{), }\DataTypeTok{sep=}\StringTok{""}\NormalTok{))  }
  \CommentTok{# re-build model with new formula}
\NormalTok{  selectedModel <-}\StringTok{ }\KeywordTok{lm}\NormalTok{(myFormula, }\DataTypeTok{data =}\NormalTok{ PhysicalMeasures)  }
\NormalTok{  all_vifs <-}\StringTok{ }\NormalTok{car}\OperatorTok{::}\KeywordTok{vif}\NormalTok{(selectedModel)}
\NormalTok{\}}
\KeywordTok{summary}\NormalTok{(selectedModel)}
\end{Highlighting}
\end{Shaded}

\subsection{Recursively select models with VIF \textless{}
4}\label{recursively-select-models-with-vif-4-1}

\begin{verbatim}
## 
## Call:
## lm(formula = myFormula, data = PhysicalMeasures)
## 
## Residuals:
##     Min      1Q  Median      3Q     Max 
## -5.7283 -1.6711  0.6758  1.4648  4.1414 
## 
## Coefficients:
##             Estimate Std. Error t value Pr(>|t|)    
## (Intercept) -83.5633    31.9510  -2.615   0.0187 *  
## Waist         0.7650     0.1241   6.166 1.36e-05 ***
## Height        0.2622     0.1079   2.431   0.0272 *  
## Calf          1.1001     0.4115   2.673   0.0167 *  
## Thigh         0.7004     0.2895   2.419   0.0278 *  
## Head         -0.5280     0.5093  -1.037   0.3152    
## ---
## Signif. codes:  
## 0 '***' 0.001 '**' 0.01 '*' 0.05 '.' 0.1 ' ' 1
## 
## Residual standard error: 2.781 on 16 degrees of freedom
## Multiple R-squared:  0.951,  Adjusted R-squared:  0.9357 
## F-statistic: 62.08 on 5 and 16 DF,  p-value: 6.591e-10
\end{verbatim}

\subsection{P-value}\label{p-value}

The p-value for Head is large, so not significant

\begin{Shaded}
\begin{Highlighting}[]
\NormalTok{car}\OperatorTok{::}\KeywordTok{vif}\NormalTok{(selectedModel)}
\end{Highlighting}
\end{Shaded}

\begin{verbatim}
##    Waist   Height     Calf    Thigh     Head 
## 2.540138 1.214250 2.687197 2.940825 1.111665
\end{verbatim}

May want to leave it in to support the model

\subsection{Recursively Remove Non-significant predictors with VIFs
Criteria}\label{recursively-remove-non-significant-predictors-with-vifs-criteria}

\begin{Shaded}
\begin{Highlighting}[]
\NormalTok{all_vars <-}\StringTok{ }\KeywordTok{names}\NormalTok{(selectedModel[[}\DecValTok{1}\NormalTok{]])[}\OperatorTok{-}\DecValTok{1}\NormalTok{]  }\CommentTok{# names of all X variables}
\CommentTok{# Get the non-significant vars}
\NormalTok{summ <-}\StringTok{ }\KeywordTok{summary}\NormalTok{(selectedModel)  }\CommentTok{# model summary}
\NormalTok{pvals <-}\StringTok{ }\NormalTok{summ[[}\DecValTok{4}\NormalTok{]][, }\DecValTok{4}\NormalTok{]  }\CommentTok{# get all p values}
\CommentTok{# init variables that aren't statistically significant}
\NormalTok{not_significant <-}\StringTok{ }\KeywordTok{character}\NormalTok{() }
\NormalTok{not_significant <-}\StringTok{ }\KeywordTok{names}\NormalTok{(}\KeywordTok{which}\NormalTok{(pvals }\OperatorTok{>}\StringTok{ }\FloatTok{0.1}\NormalTok{))}
\CommentTok{# remove 'intercept'. Optional!}
\NormalTok{not_significant <-}\StringTok{ }\NormalTok{not_significant[}\OperatorTok{!}\NormalTok{not_significant }\OperatorTok\StringTok{ "(Intercept)"}\NormalTok{] }
\CommentTok{# If there are any non-significant variables, }
\ControlFlowTok{while}\NormalTok{(}\KeywordTok{length}\NormalTok{(not_significant) }\OperatorTok{>}\StringTok{ }\DecValTok{0}\NormalTok{)\{}
\NormalTok{  all_vars <-}\StringTok{ }\NormalTok{all_vars[}\OperatorTok{!}\NormalTok{all_vars }\OperatorTok\StringTok{ }\NormalTok{not_significant[}\DecValTok{1}\NormalTok{]]}
  \CommentTok{# new formula}
\NormalTok{  myFormula <-}\StringTok{ }\KeywordTok{as.formula}\NormalTok{(}\KeywordTok{paste}\NormalTok{(}\StringTok{"Mass ~ "}\NormalTok{, }
    \KeywordTok{paste}\NormalTok{ (all_vars, }\DataTypeTok{collapse =} \StringTok{" + "}\NormalTok{), }\DataTypeTok{sep =} \StringTok{""}\NormalTok{))}
  \CommentTok{# re-build model with new formula}
\NormalTok{  selectedModel <-}\StringTok{ }\KeywordTok{lm}\NormalTok{(myFormula, }\DataTypeTok{data =}\NormalTok{ PhysicalMeasures)  }
  \CommentTok{# Get the non-significant vars.}
\NormalTok{  summ <-}\StringTok{ }\KeywordTok{summary}\NormalTok{(selectedModel)}
\NormalTok{  pvals <-}\StringTok{ }\NormalTok{summ[[}\DecValTok{4}\NormalTok{]][, }\DecValTok{4}\NormalTok{]}
\NormalTok{  not_significant <-}\StringTok{ }\KeywordTok{character}\NormalTok{()}
\NormalTok{  not_significant <-}\StringTok{ }\KeywordTok{names}\NormalTok{(}\KeywordTok{which}\NormalTok{(pvals }\OperatorTok{>}\StringTok{ }\FloatTok{0.1}\NormalTok{))}
\NormalTok{  not_significant <-}\StringTok{ }\NormalTok{not_significant[}\OperatorTok{!}\NormalTok{not_significant }\OperatorTok\StringTok{ "(Intercept)"}\NormalTok{]}
\NormalTok{\}}
\KeywordTok{summary}\NormalTok{(selectedModel)}
\end{Highlighting}
\end{Shaded}

\subsection{Recursively Remove Non-significant predictors with VIFs
Criteria}\label{recursively-remove-non-significant-predictors-with-vifs-criteria-1}

\begin{verbatim}
## 
## Call:
## lm(formula = myFormula, data = PhysicalMeasures)
## 
## Residuals:
##     Min      1Q  Median      3Q     Max 
## -5.2062 -1.8569  0.1872  1.2223  5.1725 
## 
## Coefficients:
##              Estimate Std. Error t value Pr(>|t|)    
## (Intercept) -110.8810    18.1149  -6.121 1.13e-05 ***
## Waist          0.7460     0.1230   6.066 1.26e-05 ***
## Height         0.2582     0.1080   2.391   0.0287 *  
## Calf           1.0623     0.4108   2.586   0.0192 *  
## Thigh          0.7084     0.2900   2.442   0.0258 *  
## ---
## Signif. codes:  
## 0 '***' 0.001 '**' 0.01 '*' 0.05 '.' 0.1 ' ' 1
## 
## Residual standard error: 2.787 on 17 degrees of freedom
## Multiple R-squared:  0.9477, Adjusted R-squared:  0.9354 
## F-statistic:    77 on 4 and 17 DF,  p-value: 1.161e-10
\end{verbatim}

\subsection{Forward Stepwise}\label{forward-stepwise}

\begin{Shaded}
\begin{Highlighting}[]
\KeywordTok{library}\NormalTok{(MASS); }\KeywordTok{library}\NormalTok{(caret)}
\NormalTok{res.lm <-}\StringTok{ }\KeywordTok{lm}\NormalTok{(Mass }\OperatorTok{~}\NormalTok{., }\DataTypeTok{data =}\NormalTok{ PhysicalMeasures)}
\NormalTok{step <-}\StringTok{ }\KeywordTok{stepAIC}\NormalTok{(res.lm, }\DataTypeTok{direction =} \StringTok{"forward"}\NormalTok{, }\DataTypeTok{trace =} \OtherTok{FALSE}\NormalTok{)}
\NormalTok{step}
\CommentTok{# Train the model}
\NormalTok{train.control <-}\StringTok{ }\KeywordTok{trainControl}\NormalTok{(}\DataTypeTok{method =} \StringTok{"cv"}\NormalTok{, }\DataTypeTok{number =} \DecValTok{10}\NormalTok{)}
\NormalTok{step.model <-}\StringTok{ }\KeywordTok{train}\NormalTok{(Mass }\OperatorTok{~}\NormalTok{., }\DataTypeTok{data =}\NormalTok{ PhysicalMeasures,}
                    \DataTypeTok{method =} \StringTok{"lmStepAIC"}\NormalTok{, }
                    \DataTypeTok{trControl =}\NormalTok{ train.control,}
                    \DataTypeTok{trace =} \OtherTok{FALSE}
\NormalTok{)}
\CommentTok{# Model accuracy}
\NormalTok{step.model}\OperatorTok{$}\NormalTok{results}
\CommentTok{# Final model coefficients}
\NormalTok{step.model}\OperatorTok{$}\NormalTok{finalModel}
\CommentTok{# Summary of the model}
\KeywordTok{summary}\NormalTok{(step.model}\OperatorTok{$}\NormalTok{finalModel)}
\end{Highlighting}
\end{Shaded}

\subsection{Forward Stepwise}\label{forward-stepwise-1}

\begin{verbatim}
## 
## Call:
## lm(formula = .outcome ~ Fore + Waist + Height + Calf + Thigh + 
##     Head, data = dat)
## 
## Residuals:
##     Min      1Q  Median      3Q     Max 
## -3.2362 -1.3426 -0.0132  0.9784  4.5197 
## 
## Coefficients:
##              Estimate Std. Error t value Pr(>|t|)    
## (Intercept) -79.72624   23.88925  -3.337  0.00450 ** 
## Fore          1.79485    0.48536   3.698  0.00215 ** 
## Waist         0.65671    0.09719   6.757 6.45e-06 ***
## Height        0.25388    0.08059   3.150  0.00661 ** 
## Calf          0.50718    0.34671   1.463  0.16415    
## Thigh         0.43298    0.22801   1.899  0.07698 .  
## Head         -0.65722    0.38200  -1.720  0.10590    
## ---
## Signif. codes:  
## 0 '***' 0.001 '**' 0.01 '*' 0.05 '.' 0.1 ' ' 1
## 
## Residual standard error: 2.077 on 15 degrees of freedom
## Multiple R-squared:  0.9744, Adjusted R-squared:  0.9641 
## F-statistic:    95 on 6 and 15 DF,  p-value: 4.501e-11
\end{verbatim}

\subsection{Lasso}\label{lasso}

\begin{Shaded}
\begin{Highlighting}[]
\KeywordTok{library}\NormalTok{(ISLR); }\KeywordTok{library}\NormalTok{(glmnet); }\KeywordTok{library}\NormalTok{(dplyr); }\KeywordTok{library}\NormalTok{(tidyr)}
\NormalTok{X =}\StringTok{ }\NormalTok{PhysicalMeasures[}\OperatorTok{-}\DecValTok{1}\NormalTok{] }\OperatorTok\StringTok{ }\KeywordTok{as.matrix}\NormalTok{()}
\NormalTok{y =}\StringTok{ }\NormalTok{PhysicalMeasures[,}\DecValTok{1}\NormalTok{] }\OperatorTok\StringTok{ }\KeywordTok{as.matrix}\NormalTok{()}

\CommentTok{#must specify alpha=1 for lasso}
\NormalTok{fit_lasso =}\StringTok{ }\KeywordTok{cv.glmnet}\NormalTok{(X, y, }\DataTypeTok{alpha=}\DecValTok{1}\NormalTok{)}
\CommentTok{#we can obtain the lasso estimates}
\KeywordTok{predict}\NormalTok{(fit_lasso, }\DataTypeTok{s =}\StringTok{"lambda.min"}\NormalTok{, }\DataTypeTok{type =} \StringTok{"coefficients"}\NormalTok{)}
\end{Highlighting}
\end{Shaded}

\subsection{Lasso}\label{lasso-1}

\begin{verbatim}
## Warning: Option grouped=FALSE enforced in cv.glmnet, since <
## 3 observations per fold
\end{verbatim}

\begin{verbatim}
## 11 x 1 sparse Matrix of class "dgCMatrix"
##                        1
## (Intercept) -82.78478083
## Fore          1.55474045
## Bicep         .         
## Chest         0.07852222
## Neck          .         
## Shoulder      0.04207478
## Waist         0.62789420
## Height        0.23686304
## Calf          0.53379017
## Thigh         0.36340188
## Head         -0.56824246
\end{verbatim}

\subsection{Elastic Net}\label{elastic-net}

\begin{Shaded}
\begin{Highlighting}[]
\CommentTok{#must specify alpha=1 for lasso}
\NormalTok{fit_elnet <-}\StringTok{ }\KeywordTok{cv.glmnet}\NormalTok{(X, y, }\DataTypeTok{alpha=}\NormalTok{.}\DecValTok{5}\NormalTok{,)}
\KeywordTok{predict}\NormalTok{(fit_elnet, }\DataTypeTok{s =}\StringTok{"lambda.min"}\NormalTok{, }\DataTypeTok{type =} \StringTok{"coefficients"}\NormalTok{)}
\end{Highlighting}
\end{Shaded}

\subsection{Elastic Net}\label{elastic-net-1}

\begin{verbatim}
## Warning: Option grouped=FALSE enforced in cv.glmnet, since <
## 3 observations per fold
\end{verbatim}

\begin{verbatim}
## 11 x 1 sparse Matrix of class "dgCMatrix"
##                       1
## (Intercept) -97.4622343
## Fore          1.1793938
## Bicep         .        
## Chest         .        
## Neck          .        
## Shoulder      0.1875325
## Waist         0.5442076
## Height        0.1549015
## Calf          0.5727904
## Thigh         0.4437758
## Head          .
\end{verbatim}

\subsection{Chosen Model: Parsimonious}\label{chosen-model-parsimonious}

\begin{Shaded}
\begin{Highlighting}[]
\KeywordTok{summary}\NormalTok{(selectedModel)}
\end{Highlighting}
\end{Shaded}

\begin{verbatim}
## 
## Call:
## lm(formula = myFormula, data = PhysicalMeasures)
## 
## Residuals:
##     Min      1Q  Median      3Q     Max 
## -5.2062 -1.8569  0.1872  1.2223  5.1725 
## 
## Coefficients:
##              Estimate Std. Error t value Pr(>|t|)    
## (Intercept) -110.8810    18.1149  -6.121 1.13e-05 ***
## Waist          0.7460     0.1230   6.066 1.26e-05 ***
## Height         0.2582     0.1080   2.391   0.0287 *  
## Calf           1.0623     0.4108   2.586   0.0192 *  
## Thigh          0.7084     0.2900   2.442   0.0258 *  
## ---
## Signif. codes:  
## 0 '***' 0.001 '**' 0.01 '*' 0.05 '.' 0.1 ' ' 1
## 
## Residual standard error: 2.787 on 17 degrees of freedom
## Multiple R-squared:  0.9477, Adjusted R-squared:  0.9354 
## F-statistic:    77 on 4 and 17 DF,  p-value: 1.161e-10
\end{verbatim}

\end{document}
